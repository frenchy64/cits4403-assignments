\documentclass{article}

\usepackage[style=alphabetic, sorting=nyt]{biblatex}
\addbibresource{bibliography.bib}

\usepackage{graphicx}
\usepackage{epstopdf}
\usepackage{color}

\graphicspath{{../images/}}


\title{CITS4403 Assignment 2 - Percolation and forest fires}
\author{Ambrose Bonnaire-Sergeant - 20350292}

\begin{document}

\maketitle
\pagebreak

% Why was the study undertaken? What was the research question, the tested hypothesis or the purpose of the research?
\section{Introduction}

% Why? What?
% Forest fire simulation
% - first introduced by Bak90, speculating that the model displays self-organised criticality
% - further investigation by Grass

% Percolation
% what are CA's for?
% what is the forest fire model? what does it achieve?
% - good description in 2nd paper: probabilistic CA
% - spatial distribution of dissipation (fire) and its dependence
%   on the driving force.
% - shows that ``energy'', when injected uniformly (trees grow uniformly), is
%   dissipated (trees burn) on a fractal.
% - demonstrates (in a toy model) the emergence of scaling and fractal energy
%   dissipation.
% - could provide a consistent and theoretically sustainable phenomenological 
%   picture of a class of turbulent phenomena

% self-organised criticality
% - see intro to Drossel for good history
% - forest fires shown to not be critical but istead 


% Rules of the forest fire model
% - parameters
% - dimensions

% When, where, and how was the study done? What materials were used or who was included in the study groups (patients, etc.)?
\section{Methods}

\subsection{Percolation Theshold}

By isolating the variables related to generating the initial lattice, we observe
the relationship between the initial lattice and the ability for a forest
fire to percolate. Setting \emph{p} = 0 and \emph{f} = 0 results in a static lattice
over time, except for the effects of fires already present at \emph{t} = 0:
these fires propagate as normal, but under these conditions
no new fires can appear spontaneously, nor can new trees grow.

\begin{figure}
  \input{../images/initial-state-q0-3}
\end{figure}

\begin{figure}
  \input{../images/initial-state-q0-4}
\end{figure}

\begin{figure}
  \input{../images/initial-state-q0-5}
\end{figure}

\begin{figure}
  \resizebox{1.3\textwidth}{!}{\input{../images/multi-nburning}}
\end{figure}

\begin{figure}
  \resizebox{1.3\textwidth}{!}{\input{../images/multi-ntrees}}
\end{figure}

\begin{figure}
  \resizebox{1.3\textwidth}{!}{\input{../images/multi-nempty}}
\end{figure}


% What answer was found to the research question; what did the study find? Was the tested hypothesis true?
\section{Results}


% What might the answer infer and why does it matter? How does it fit in with what other researchers have found? What are the perspectives for future research?
\section{Discussion}

The study of percolation in models

\end{document}
